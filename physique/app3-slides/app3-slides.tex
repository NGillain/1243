\documentclass[pdf]{beamer} 

\usepackage[utf8]{inputenc} 
\usepackage{graphicx} 
\usepackage[squaren, Gray]{SIunits} 
\usepackage{amsmath}  

\usetheme{warsaw} 
\mode<presentation>{} 
 
\title{Physique} 
\subtitle{APP 3 : Rayonnement électromagnétique} 
\author{Groupe 1243} 
 
\begin{document} 
 
\begin{frame} 
	\titlepage 
\end{frame}

\begin{frame}{Courant alternatif dans une antenne}

\end{frame} 
 
\begin{frame}{Estimation du gain en intensité rayonnée}

Supposons une antenne demi onde et une antenne classique rayonnant de manière isotrope.

On cherche le gain en intensité rayonné par l'antenne demi onde par rapport à l'antenne classique.

Intensité rayonnée pour une source ponctuelle:

$$ I = \epsilon_{0}\cdot v\cdot E^{2} $$

\end{frame}
 
\end{document}

\input{../lib.tex}
\doctitle{Étude de l'impact environnemental}
Cette tâche a pour but de citer et dimensionner approximativement l'empreinte écologique du procédé. Nous commencerons par donner les consommations énergétiques de chaque parties de celui-ci puis nous continuerons avec les rejets de déchets (sous-entendu substances qui sont produites et inutiles dans la production) et enfin nous citerons quelques pistes afin d'améliorer le procédé.

\section{Dimensionnement de la consommation énergétique}
\label{dim}
Afin de trouver la consommation énergétique du procédé, on cherche l'enthalpie d'une réaction à la température donné puis on la multiplié par le nombre de moles en quantité stoechiométrique. Nous avons pris des hypothèses pour simplifier le modèle et calculer une approxiamtion plus facile à obtenir(mais toujours utile). Nous avons donc considérer que :
\begin{itemize}
	\item On suppose les matières premières pures et sans poisons catalytiques(souffre)
	\item On considère le système en régime
  \item Les capacités calorifiques sont des polynômes de température et varient donc en fonction de celle-ci.
	\item L'énergie nécessaire pour la séparation entre le CO_{2} et l'eau est négligeable,
	\item Seul 75\% de l'énergie produite par le four est réellement utilisée pour chauffer le reformage primaire
	\item Toutes les réactions sauf celles du reformage primaire sont complètes (ou quasi-complètes)
	\item La température du four est de \unit{1000}{\kelvin}
	\item La production de NH_{3} est de \unit{1500}{\tons\per\dday}
\end{itemize}
	
Pour toutes ces hypothèses, voici les différentes consommations énergétiques des parties de la chaîne de production (dans l'ordre):
\begin{itemize}
	\item Reformage primaire(RF1): \unit{36343.48}{\kilo\joules}\\
	\item Four du RF1 : \unit{-48457.49}{\kilo\joules}\\
	\item Reformage secondaire(RF2) : \unit{-5578.73}{\kilo\joules}\\
	\item Water-Gas_Shift(WGS) : \unit{-19602.05}{\kilo\joules}\\
	\item Séparation : \unit{0}{\kilo\joules}\\
	\item Synthèse : \unit{-28573.18}{\kilo\joules}\\
\end{itemize}
Nous remarquons que la partie la plus énergivore du procédé et la seule à demander de l'énergie est le réformage primaire. C'est pourquoi le four est nécessaire pour favoriser les réactions qui produisent du dihydrogène.

\section{Dimensionnement des productions de produits secondaires}

Ensuite, vient les production de produits non-désirés que nous devrons traités ou relâcher dans la nature s'ils ne sont pas toxiques. Voici donc les productions des divers produits en commençant par le dioxyde de carbone.

\subsection{Le dioxyde de carbone}

Les hypothèses dont identiques à celles formulées dans la section du dimensionnement énergétique\ref{dim}. Voici donc les productions de CO_{2} par seconde:

\begin{itemize}
	\item Reformage primaire(RF1): \unit{99.34}{\mol}\\
	\item Four du RF1 : \unit{60.12}{\mol}\\
	\item Reformage secondaire(RF2) : \unit{0}{\mol}\\
	\item Water-Gas_Shift(WGS) : \unit{352.36}{\mol}\\
	\item Séparation : \unit{0}{\mol}\\
	\item Synthèse : \unit{0}{\mol}\\
\end{itemize}

Nous observons que cette fois-ci la réaction la plus polluante est le "shift conversion". Elle est malheureusement la plus usitée pour la production de dihydrogène. Notons que l'énergie du four peut provenir d'une autre source que la combustion de méthane.

\subsection{Argon et autres produits}

Ici, nous parlerons des autres produits comme l'argon ou les oxydes d'azotes produit dans le four par le combustion dans un espace fermé de méthane. L'argon provient de l'air et sa concentration est élevée est due à une hypothèse (très) simplificatrice : l'air est composé de 78\% de diazote, de 21\% de dioxygène et de 1\% d'argon. Donc, nous avons que
\begin{itemize}
	\item nombre de moles d'argon présent dans le système : \unit{6.55}{\mol\per\second}
	\item masse de NO_{x} (et surtout de NO_{2}) présente : entre 0.6 et \unit{1.3}{\tons\per\second}
\end{itemize}

\section{Pistes pour améliorer l'empreinte écologique du procédé}
%
% http://www.lenntech.fr/data-perio/n.html
\biblio{sources-tache3}
\end{document}
